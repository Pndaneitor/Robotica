\section{Valeria Cancio}
En este proyecto se trabajó con un robot manipulador de 4 grados de libertad, diseñado en SolidWorks y simulado en ROS sobre Ubuntu. Se hicieron los análisis de cinemática directa, inversa y diferencial usando MATLAB, lo que permitió entender cómo se mueve el robot, cómo calcular sus posiciones y cómo seguir trayectorias definidas.

Se implementaron trayectorias suaves con perfiles trapezoidales y se validaron los movimientos con animaciones y gráficas. Además, el robot fue construido con materiales adecuados como ABS y eslabones metálicos, y se usaron distintos tipos de motores según las necesidades de cada articulación.

Me gustó mucho trabajar con MATLAB, ya que fue interesante ver cómo, a través de este programa, podíamos controlar y simular el movimiento del robot. Me ayudó a comprender de una forma más visual y práctica cómo aplicar los conceptos de la robótica.

En general, este proyecto ayudó a comprender mejor cómo funcionan los robots y cómo aplicar la teoría a un sistema real.
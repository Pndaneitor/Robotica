\section{Yairee Arredondo}

Durante este proyecto se logró modelar, simular y establecer los límites y parámetros de un robot con 4 grados de libertad con el uso de diversas nuevas herramientas. Conforme se fueron desarrollando los temas se fueron integrando nuevas herramientas y conocimientos que serán de gran utilidad en futuros proyectos. 
El aprender y trabajar con nuevos softwares me brindó un panorama más amplio y contribuyó a mi desarrollo, ya que podré contar con esas herramientas tales como Latex, Github, y Matlab de manera más profunda. 
A lo largo del desarrollo del curso se estuvieron viendo temas teoricos que al final se pudieron implementar de manera práctica y lograron concretarse en un robot capaz de completar determinada taryectoria.
Personalmente conté con los retos de aprender a manejar programas y softwares desconocidos y a tener problemas con mi equipo de cómputo  pero mediante el trabajo en equipo pudimos sobrellevar dichas situaciones y realizar de manera exitosa el proyecto solicitado. 

Considero muy interesante el conocer todo el transfondo y los conocimientos que se adquieren al programar, modelar y delimitar las articulaciones de un robot y la robótica es un campo que sin duda alguna quisiera seguir explorando.
\section{Diego Vásquez}
Personalmente, me divertí mucho en esta materia. Al comienzo me sentí un poco perdido, pues la verdad no comprendía del todo el por qué de lo que veíamos en clase, fue hasta que nos adentramos más en la parte práctica del robot que mis dudas fueron disipadas y por fin comprendí el propósito de la teoría. Entrando más en detalle, mi parte favorita fue todo lo de ROS. Las principales dificultades de la realización de nuestro robot, sobre todo la parte simulada, fue la poca familiaridad que se tenía con el entorno, así como lo “viejo” del tutorial, pues hubieron algunas partes que tuvimos que resolver por nuestra cuenta.

En general, fue muy entretenido simular, experimentar e incluso batallar con las irregularidades que nos iban apareciendo. Pienso que es una muy buena materia para familiarizarse un poco con la realidad de la industria, y, asimismo, aprender un poco mientras se conoce más sobre el mundo de la robótica y sus aplicaciones.
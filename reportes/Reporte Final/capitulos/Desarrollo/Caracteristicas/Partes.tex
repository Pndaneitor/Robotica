\subsubsection{Motores} \label{subsubsec:motores}

El sistema de accionamiento del robot incorpora distintos tipos de motores, seleccionados estratégicamente según los requerimientos específicos de torque, precisión y control de cada componente:

\begin{itemize}
	\item \textbf{Motores servo SureServo2 (modelo SV2M-220B):} Se utilizaron dos motores \textit{brushless} de corriente alterna, cada uno con una potencia nominal de 2 kW y un par de sujeción de 84.5 lb-in. Están equipados con encoders de alta resolución de 24 bits (16,777,216 pulsos por revolución) y freno integrado. Estos motores operan con entrada trifásica y ofrecen velocidades de hasta 3000 rpm, siendo ideales para tareas que requieren alta precisión y control dinámico. Fueron acoplados a cajas reductoras de precisión para optimizar su desempeño mecánico.
	
	\item \textbf{Motor de corriente continua Weonefit (DC 24 V, 45 W):} Este motor de engranajes proporciona una salida de 220 rpm con un par nominal de 30 Nm, siendo adecuado para aplicaciones de bajo consumo energético que requieren buen torque a velocidades moderadas. Su construcción en materiales metálicos y plásticos lo hace robusto para tareas de baja complejidad mecánica.
	
	\item \textbf{Motor paso a paso STEPPERONLINE con caja reductora planetaria 47:1:} Este conjunto incorpora un motor NEMA 23 bipolar con un torque de sujeción de hasta 40 Nm y una corriente nominal de 2.8 A por fase. Gracias a la caja planetaria, se obtiene una alta relación de reducción, lo que permite movimientos controlados, precisos y con alto torque, útiles para tareas que no demandan retroalimentación continua.
\end{itemize}


\subsubsection{Eslabones} \label{subsubsec:eslabones}

La estructura mecánica del robot está compuesta por eslabones metálicos que forman los brazos y segmentos articulados. Cada eslabón fue diseñado considerando tanto la distribución de masas como la longitud efectiva de trabajo, asegurando estabilidad dinámica y resistencia estructural. A continuación se describen sus características principales:

\begin{itemize}

	\item \textbf{Eslabón 1:} Masa de 17.68 kg, longitud de 0.28 m.
	\item \textbf{Eslabón 2:} Masa de 15.5 kg, longitud de 0.30 m.
	\item \textbf{Eslabón 3:} Masa de 6.5 kg, longitud de 0.25 m.
	\item \textbf{Eslabón 4:} Masa de 6.5 kg, longitud de 0.35 m.
\end{itemize}

Estas dimensiones fueron seleccionadas considerando el equilibrio entre la carga que deben soportar, la eficiencia en la transmisión del movimiento y la minimización del momento de inercia en cada articulación. Además, su disposición permite al robot alcanzar posiciones complejas y ejecutar trayectorias con precisión y estabilidad.

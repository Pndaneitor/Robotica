\section{Proceso de Cinemática} \label{sec:proceso_cinematica}

Aquí explicarán su código. Si quieren mostrar una parte, pueden hacerlo de la siguiente forma

% Si no quieres ponerle título al código, puedes dejarlo en blanco.
\begin{matlabcode}{matlab}
	function [q_sol, p_sol] = cinematica_inv(r, p_des, tol, max_iter, alpha, numMuestras)
\end{matlabcode}

Pero solo háganlo en partes muy específicas (las que van a explicar en ese momento). No copien todo el código ya que eso está en GitHub.

Si les sale el error \texttt{latexminted no se reconoce como un comando interno o externo, programa o archivo por lotes ejecutable}, deben tener instalado python y usar el siguiente comando.
\begin{terminal}{bash: Instalar minted en python con pip}
	pip install latexminted==0.5.1
\end{terminal}

\subsection{Cinemática Directa}
Explicar las partes importantes del código de la cinemática directa.
Para ver los resultados, ir al \autoref{chap:resultados}: Resultados, o determinada figura.
\subsection{Cinemática Diferencial}
Explicar las partes importantes del código de la cinemática diferencial.
Para ver los resultados, ir al \autoref{chap:resultados}: Resultados, o determinada figura.
\subsection{Cinemática Inversa}
Explicar las partes importantes del código de la cinemática inversa.
Para ver los resultados, ir al \autoref{chap:resultados}: Resultados, o determinada figura.
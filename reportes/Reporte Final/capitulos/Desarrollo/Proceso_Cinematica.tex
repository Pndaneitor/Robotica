\section{Proceso de Cinemática} \label{sec:proceso_cinematica}

\subsection{Cinemática Directa}

La cinemática directa permite determinar la posición y orientación del efector final del robot a partir de los valores articulares. A continuación, se explican las secciones más relevantes del código empleado.

\subsubsection*{Creación de la estructura del robot}

Se leyó la tabla DH del robot desde un archivo CSV, y se construyó su modelo mediante la función \texttt{crear\_robot}, que genera las matrices de transformación homogénea necesarias para cada eslabón.

\begin{matlabcode}{matlab}
	dh = readtable('datos\tabla_DH\robotnuestro.csv');
	robot = crear_robot(dh, A0);
\end{matlabcode}

\subsubsection*{Generación de trayectorias articulares}

Para simular el movimiento del robot, se generaron trayectorias para las posiciones (\texttt{q}), velocidades (\texttt{dq}) y aceleraciones (\texttt{ddq}) articulares. Estas se utilizaron como entrada para los cálculos de cinemática.

\begin{matlabcode}{matlab}
	[q, dq, ddq] = trayectoria_q(robot, t, periodo);
\end{matlabcode}

\subsubsection*{Cálculo de posición y orientación del efector}

Se recorrió cada instante de tiempo para actualizar la configuración del robot, obteniendo así la posición y orientación del efector final en el espacio. La orientación se calculó en ángulos de Euler a partir de la matriz de rotación.

\begin{matlabcode}{matlab}
	for k = 1:length(t)
	robot = actualizar_robot(robot, q(:,k));
	posicion(:,k) = robot.T(1:3,4,end);
	R(:,:,k) = robot.T(1:3,1:3,end);
	orientacion(:,k) = rotMat2euler(R(:, :, k), secuencia);
	end
\end{matlabcode}

\subsubsection*{Cálculo de velocidades y aceleraciones}

Usando el Jacobiano geométrico, se calcularon la velocidad y aceleración lineal y angular del efector final. Se aplicó la fórmula:

\[
\dot{x} = J(q)\,\dot{q}, \quad \ddot{x} = J(q)\,\ddot{q} + \dot{J}(q)\,\dot{q}
\]

\begin{matlabcode}{matlab}
	vel_linear(:,k)  = Jv(:,:,k) * dq(:,k); 
	vel_angular(:,k) = Jw(:,:,k) * dq(:,k);  
	
	acel_linear(:,k)  = Jv(:,:,k) * ddq(:,k) + dJv(:,:,k) * dq(:,k); 
	acel_angular(:,k) = Jw(:,:,k) * ddq(:,k) + dJw(:,:,k) * dq(:,k);
\end{matlabcode}

\subsubsection*{Visualización}

Se generó una animación para validar visualmente el movimiento del robot. Esta animación fue útil para corroborar que la cinemática directa estaba funcionando correctamente.

\begin{matlabcode}{matlab}
	g = crear_grafica_robot();
	for k = 1:length(t_animacion)
	robot = actualizar_robot(robot, q_anim(:,k));
	g = dibujar_robot(g, robot);
	end
\end{matlabcode}

Las gráficas de resultados correspondientes a posición, orientación, velocidad y aceleración se presentan en la sección de \autoref{chap:resultados}.

\subsection{Cinemática Diferencial}

La cinemática diferencial permite calcular la velocidad y aceleración del efector final a partir de las velocidades y aceleraciones articulares, usando el Jacobiano del robot. A continuación, se presenta la implementación en MATLAB dividida en pasos relevantes.

\subsubsection*{Actualización de la configuración del robot}

Para cada instante de tiempo, se actualizó el modelo del robot con los valores articulares correspondientes.

\begin{matlabcode}{matlab}
	for k = 1:length(t)
	robot = actualizar_robot(robot, q(:,k));
\end{matlabcode}

\subsubsection*{Extracción de posición y orientación}

Se extrajo la posición y la matriz de rotación del efector final, y luego se convirtieron en ángulos de Euler para analizar la orientación.

\begin{matlabcode}{matlab}
	posicion(:,k)   = robot.T(1:3,4,end);
	R(:,:,k)        = robot.T(1:3,1:3,end);
	orientacion(:,k)= rotMat2euler(R(:,:,k), secuencia);
\end{matlabcode}

\subsubsection*{Cálculo del Jacobiano geométrico}

Se obtuvo el Jacobiano geométrico, que relaciona las velocidades articulares con las velocidades del efector final.

\begin{matlabcode}{matlab}
	[Jv(:,:,k), Jw(:,:,k)] = jac_geometrico(robot);
\end{matlabcode}

\subsubsection*{Cálculo de velocidades del efector final}

A partir del Jacobiano y de las velocidades articulares se obtuvieron la velocidad lineal y angular del efector final.

\begin{matlabcode}{matlab}
	vel_linear(:,k)  = Jv(:,:,k) * dq(:,k);
	vel_angular(:,k) = Jw(:,:,k) * dq(:,k);
\end{matlabcode}

\subsubsection*{Derivada del Jacobiano (diferencias finitas)}

Para calcular las aceleraciones se necesitó la derivada del Jacobiano, la cual se obtuvo mediante diferencias finitas.

\begin{matlabcode}{matlab}
	if k > 1
	dJv(:,:,k) = (Jv(:,:,k) - Jv(:,:,k-1)) / dt;
	dJw(:,:,k) = (Jw(:,:,k) - Jw(:,:,k-1)) / dt;
	end
\end{matlabcode}

\subsubsection*{Cálculo de aceleraciones del efector final}

Finalmente, se calcularon las aceleraciones lineales y angulares del efector usando la fórmula completa de la cinemática diferencial:

\[
\dot{x} = J(q)\,\dot{q}, \quad \ddot{x} = J(q)\,\ddot{q} + \dot{J}(q)\,\dot{q}
\]

\begin{matlabcode}{matlab}
	acel_linear(:,k)  = Jv(:,:,k) * ddq(:,k) + dJv(:,:,k) * dq(:,k);
	acel_angular(:,k) = Jw(:,:,k) * ddq(:,k) + dJw(:,:,k) * dq(:,k);
	end
\end{matlabcode}

\subsubsection*{Resultados}

Los datos obtenidos fueron utilizados para generar gráficas de velocidad y aceleración tanto lineal como angular, así como para validar el comportamiento dinámico del efector final durante la trayectoria.



\subsection{Cinemática Inversa}

La cinemática inversa permite encontrar los valores articulares que lleva al efector final a una posición deseada en el espacio. A continuación, se describen las secciones más relevantes del código empleado.

\subsubsection*{Definición de la posición y orientación inicial}

Se definieron los valores iniciales de posición y orientación del efector final, así como la matriz de transformación homogénea inicial.

\begin{matlabcode}{matlab}
	x = 0; y = 0; z = 0;
	p_robot = [x; y; z];
	phi = 0; theta = 0; psi = 0;
	ori_robot = deg2rad([phi; theta; psi]);
	R_inicial = euler2rotMat(ori_robot, "XYZ");
	A0 = [R_inicial p_robot; zeros(1,3) 1];
\end{matlabcode}

\subsubsection*{Creación del modelo del robot}

Se leyó la tabla DH desde un archivo y se construyó el modelo del robot.

\begin{matlabcode}{matlab}
	dh = readtable('datos\tabla_DH\robotnuestro.csv');
	robot = crear_robot(dh, A0);
\end{matlabcode}


\subsubsection*{Resolución de la cinemática inversa}

Se empleó un método iterativo basado en descenso del gradiente para minimizar el error entre la posición deseada y la alcanzada. En caso de quedar atrapado en un mínimo local, se evaluaron múltiples candidatos articulares.

\begin{matlabcode}{matlab}
	tol = 1e-6;
	max_iter = 100;
	alpha = 1.0;
	numSamples = 9;
	
	for i = 1:n
	[q_sol(:,i), p_sol(:,i)] = cinematica_inv(robot, pos_puntos(:,i), ...
	tol, max_iter, alpha, numSamples);
	end
\end{matlabcode}

\subsubsection*{Generación de trayectoria articular continua}

Se interpolaron las posiciones articulares resultantes de la cinemática inversa para obtener perfiles suaves de velocidad y aceleración mediante trayectorias trapezoidales.

\begin{matlabcode}{matlab}
	delta_q = abs(q_sol(:,2) - q_sol(:,1));
	t_final = max(2 * delta_q ./ robot.dqMax);
	numSamples = 201;
	
	[q, dq, ddq, t, pp] = trapveltraj(q_sol, numSamples, ...
	'Acceleration', robot.ddqMax, ...
	'EndTime', 5);
\end{matlabcode}

\subsubsection*{Validación mediante cinemática directa}

Se aplicó cinemática directa a las trayectorias articulares interpoladas para verificar que la posición alcanzada correspondiera con la deseada.

\begin{matlabcode}{matlab}
	[posicion, vel_linear, acel_linear, orientacion, ...
	vel_angular, acel_angular] = cinematica_dir(robot, q, dq, ddq, t, "XYZ");
\end{matlabcode}

\subsubsection*{Animación}

Se generó una animación del robot en movimiento usando la configuración articular evaluada en cada instante de tiempo.

\begin{matlabcode}{matlab}
	g = crear_grafica_robot();
	for k = 1:length(t_animacion)
	for i = 1:robot.NGDL
	q_anim(i) = ppval(cell2mat(pp(i)), t_animacion(k));
	end
	robot = actualizar_robot_completo(robot, q_anim);
	g = dibujar_robot(g, robot);
	end
\end{matlabcode}


\chapter{Introducción} \label{chap:introduccion}

El desarrollo de robots manipuladores es una parte fundamental dentro de la robótica, ya que permite automatizar tareas que requieren precisión, repetibilidad y control del movimiento. En este proyecto se trabajó con un robot manipulador de tipo articulado, compuesto por cuatro grados de libertad, diseñado específicamente para simular tareas de posicionamiento y orientación en el espacio.

El diseño mecánico del robot se realizó en SolidWorks, cuidando tanto su estructura como la distribución de masas en sus eslabones. Para su simulación, se utilizó el entorno de ROS (Robot Operating System) corriendo sobre Ubuntu 20.04 mediante WSL y Visual Studio. Además, se integraron distintos motores, como servos, motores DC y motores paso a paso con caja reductora, para lograr un movimiento más preciso y realista.

Durante el desarrollo del proyecto, se utilizaron herramientas de programación en MATLAB para analizar el comportamiento del robot a través de su cinemática directa, inversa y diferencial. Esto permitió simular el movimiento del efector final, calcular trayectorias y validar que el robot cumplía con los objetivos planteados. Esta experiencia no solo fue útil para aplicar conocimientos teóricos, sino también para aprender a controlar un robot desde el punto de vista computacional y práctico.


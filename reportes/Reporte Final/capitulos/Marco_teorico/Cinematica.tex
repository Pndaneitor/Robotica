\section{Cinemática} \label{sec:cinematica}

La cinemática es una rama de la física que estudia el movimiento de los objetos sólidos en función del tiempo, sin considerar las causas (fuerzas) que lo originan. Este análisis se basa en variables como la posición, la velocidad y la aceleración, y considera tres elementos fundamentales: el espacio, el tiempo y el móvil. En el contexto de la robótica, se aplica especialmente a sistemas de cuerpos rígidos como los manipuladores, describiendo sus trayectorias sin involucrar dinámicas.

En robótica, la cinemática se clasifica principalmente en dos tipos: \textbf{cinemática directa} y \textbf{cinemática inversa}. La primera permite determinar la posición y orientación del efector final a partir de los valores articulares del robot, mientras que la segunda se enfoca en calcular los valores articulares que permiten alcanzar una posición y orientación deseadas del efector final.

\subsection{Cinemática Directa}

La cinemática directa (o directa geométrica) establece la relación entre los ángulos articulares del robot y la posición y orientación del efector final. Para esta tarea, comúnmente se emplea el método de Denavit-Hartenberg (DH), que permite representar de forma sistemática la configuración espacial del robot mediante transformaciones homogéneas.

Dichas transformaciones permiten obtener no solo la ubicación del efector final, sino también su orientación en el espacio. Además, conceptos como el jacobiano geométrico son fundamentales para analizar la velocidad y aceleración del sistema en función de sus variables articulares.

Un desarrollo detallado del análisis computacional aplicado a este modelo puede encontrarse en la \hyperref[sec:proceso_cinematica]{sección del proceso de Cinemática}.


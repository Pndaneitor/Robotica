\chapter{Marco Teórico}
\label{chap:marco_teorico}

En este capítulo se presenta el marco teórico que respalda el proyecto de robótica. Se explican tres temas principales: la \textit{Cinemática}, la \textit{Dinámica} y el uso del sistema operativo para robots conocido como \textit{ROS}. Estos conceptos son clave para entender cómo funcionan y se controlan los manipuladores robóticos. Por un lado, la cinemática nos permite analizar el movimiento del robot sin tener en cuenta las fuerzas que lo provocan, mientras que la dinámica sí considera esas fuerzas y cómo afectan al sistema. En cuanto a ROS, se trata de una herramienta muy utilizada en el desarrollo y simulación de robots, ya que permite conectar diferentes partes del sistema como sensores, motores y programas de control de manera organizada. En las siguientes secciones se hablará brevemente sobre cómo cada uno de estos temas aporta al desarrollo y simulación de un robot manipulador. Las fórmulas necesarias para este análisis se pueden revisar en las diapositivas del curso o directamente desde los códigos de MATLAB. Además, en el caso de ROS, se incluirá una pequeña investigación en línea para entender mejor cómo está estructurado y en qué situaciones se utiliza.


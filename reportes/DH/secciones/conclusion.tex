\section{Conclusión} \label{sec:conclusion}
\vspace{10mm}
Durante la elaboración de este trabajo nos enfrentamos a diversos retos que complicaron nuestro trabajo, siendo el más grande el comprender y aprender a utilizar la herramientas de LaTex y Github. 

Al no estar familiarizados con estos entornos de programación, tuvimos problemas al comenzar a enlazarnos para colaborar en el documento sin embargo logramos aprender y realizar un buen trabajo. Otro problema que se nos presentó fue utilizar la sintaxis de LaTex, no obstante al momento de comenzar a trabajar se nos fue facilitando con la practica. 
\newpage  % Salto de página forzado
\begin{thebibliography}{99}
	\vspace{10mm}
	
	\bibitem{herga} 
	HERGA. (2020). 
	\textit{¿Qué es un interruptor neumático?}. Recuperado el 14 de febrero de 2025 de \url{https://www.herga.com/news-media/technical-blog-archive/what-is-a-pneumatic-switch-}
	
	\bibitem{kolstad} 
	Kolstad, C. (2025). 
	\textit{¿Qué es un Transductor de Presión?}. TAMESON. Recuperado el 14 de febrero de 2025 de \url{https://tameson.es/pages/transductores-de-presion-como-funcionan}
	
	\bibitem{peterson} 
	Peterson, Z. (2023). 
	\textit{Dominando el magnetismo: sensores de efecto Hall y sus aplicaciones para PCB}. Altium. Recuperado el 14 de febrero de 2025 de \url{https://resources.altium.com/es/p/mastering-magnetism-hall-effect-sensors-and-applications-pcbs}
	
	\bibitem{senstar} 
	SENSTAR. (s.f.). 
	\textit{Sensores de microondas}. Recuperado el 14 de febrero de 2025 de \url{https://senstar.com/es/senstarpedia/sensores-de-microondas/#:~:text=Los%20sensores%20de%20microondas%20utilizan,el%20principio%20del%20efecto%20Doppler}
	
	\bibitem{keyence1} 
	KEYENCE. (s.f.). 
	\textit{¿Qué es un sensor ultrasónico?}. Recuperado el 14 de febrero de 2025 de \url{https://www.keyence.com.mx/ss/products/sensor/sensorbasics/ultrasonic/info/}
	
	\bibitem{keyence2} 
	KEYENCE. (s.f.). 
	\textit{¿Qué es un sensor láser de tipo de reconocimiento de “luz recibida”?}. Recuperado el 14 de febrero de 2025 de \url{https://www.keyence.com.mx/ss/products/sensor/sensorbasics/laser_light/info/}
	
	\bibitem{keyence3} 
	KEYENCE. (s.f.). 
	\textit{¿Qué son los sensores de visión?}. Recuperado el 14 de febrero de 2025 de \url{https://www.keyence.com.mx/ss/products/sensor/sensorbasics/vision/info/}
	
	\bibitem{ridgway} 
	Ridgway, J. (2022). 
	\textit{¿Qué son los sensores de visión?}. Cognex. Recuperado el 14 de febrero de 2025 de \url{https://www.cognex.com/es-mx/blogs/machine-vision/what-are-vision-sensors}
	
	\bibitem{Saha} 
	Saha, S. K. (2008)
	\textit{Introducción a la robótica.} McGraw Hill.
	
	\bibitem{Barrientos} 
	Barrientos, A. y Peñin, L. (2007) 
	\textit{FUNDAMENTOS DE ROBÓTICA.} McGraw Hill.
	
	\bibitem{Servo} 
	Servo-user. (2025). 
	\textit{¿Cuál es la diferencia entre encoder incremental y absoluto?}Servomotors Adjust. Recuperado el 14 de febrero de 2025 de \url{https://www.servomotorsadjust.com/cual-es-la-diferencia-entre-un-encoder-incremental-y-absoluto/}

	\bibitem{Lidar} 
	IBM. (s.f.). 
	\textit{¿Qué es LiDAR?}  Recuperado el 14 de febrero de 2025 de \url{https://www.ibm.com/mx-es/topics/lidar}
	
	
\end{thebibliography}


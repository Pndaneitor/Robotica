\section{Introducción}
\vspace{10mm}
En la robótica se encuentran presentes los sensores, los cuales desempeñan un papel fundamental al proporcionar información sobre el entorno y el estado del robot, permitiendo la toma de decisiones y la ejecución de tareas con precisión. Este trabajo de investigación aborda los diferentes tipos de sensores utilizados en robótica, clasificándolos según su funcionamiento y aplicación.
Existen sensores de posición, velocidad y contacto, así como tecnologías avanzadas como sensores de proximidad, visión y LiDAR, que juegan un papel muy importante en la navegación y control de robots. Además, se presentan sensores adicionales como giroscopios y acelerómetros, que permiten medir variables físicas clave para el correcto desempeño de sistemas robóticos.
El estudio de estos sensores es de gran importancia para comprender cómo los robots pueden interactuar de manera eficiente, así como mejorar su autonomía. A lo largo de esta investigación, se describen sus características, principios de funcionamiento y aplicaciones, mostrando lo importantes que son en el área de la tecnología.

\newpage  % Salto de página forzado
